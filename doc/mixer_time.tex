\documentclass[12pt,letterpaper]{article}
\usepackage[utf8]{inputenc}
\usepackage{amsmath}
\usepackage{amssymb}

\DeclareMathOperator{\Prob}{P}
\DeclareMathOperator{\E}{E}
\DeclareMathOperator{\n}{n}
\DeclareMathOperator{\erf}{erf}

\title{Real-time event mapping}
\author{Dietrich Epp}
\date{August 7, 2014}
\begin{document}
\maketitle

Real-time mixers are fed a stream of timestamped events, and must map the timestamps to sample positions in consecutive timestamped output buffers.
A naïve solution is to map each timestamp to the directly corresponding output sample position.
However, events corresponding to a buffer may be received by the time the buffer must be rendered.
A simple solution is to delay those events until they are received, but this will affect the relative timing of different events.
Another simple solution is to delay output until the events are received, but this will cause gaps in audio playback.
Our solution is to automatically adjust event delay to maintain relative timing while minimizing delay and jitter.

In a stream of events, the \textit{commit time} is the oldest timestamp at which new events may yet be received.
Events from before the commit time may be rendered to output without affecting relative timing.

Let output buffer $t \in \mathbb{Z}$ contain audio with sample positions $t - 1\le y < t$, and let $c_t$ be the current commit time when buffer $t$ is rendered.
Our goal is to create a function $f$ mapping input timestamps to sample positions, with the following properties:
\begin{itemize}
\item It is unlikely that $f(c_t) < t$, as this would mean that events received in the future should have been placed in previously rendered buffers.
\item $f$ is continuous, monotonically increasing, and piecewise linear.
\end{itemize}

\section*{Modeling and prediction}

We are lazy, so we model the commit timestamp for buffer $t$ as
\begin{equation*}
  C_t = \alpha + \beta t + \epsilon_t,
\end{equation*}
where the error $\epsilon_t$ is an iid random normal variable with variance $\sigma^2$.
The algorithm samples $N$ commit times $c_1,\dotsc,c_N$, and uses the commit times to construct the next segment of $f$.

We can fit the $(x_t, t)$ points to a linear function,
\begin{equation*}
  g(x) = a + bx,
\end{equation*}
giving parameters
\begin{align*}
  b & = \frac{\E[x_t t] - \E[x_t]\E[t]}{\E[x^2] - \E[x]^2}, \\
  a & = \E[t] - b\E[x_t],
\end{align*}
where $\E[X]$ gives the expected value of $X$ taken from the samples.

We then add a delay $\delta$ to $g$, tuned to give the desired probability $p$ that $g(x_t) + \delta < t$.
Using the sample variance as an estimate for the true variance, we choose
\begin{equation*}
  \delta = \sqrt{s_{N}^2} \erf^{-1}(p).
\end{equation*}

A point $(x, g(x)+\delta)$ is then sampled and linearly interpolated to form $f$.
These points can be filtered using an IIR filter, to reduce the effect of noise and remove sudden changes in the slope of $f$.

\end{document}

